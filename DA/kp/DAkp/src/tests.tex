\section{Пример работы и тесты}

Пусть имеется 2 файла:

\textbf{f1.txt}
\begin{verbatim}
Эта часть документа
оставалась неизменной
от версии к версии. Если
в ней нет изменений, она
не должна отображаться.
Иначе это не способствует
выводу оптимального
объёма произведённых
изменений.

Этот абзац содержит
устаревший текст.
Он будет удалён
в ближайшем будущем.

В этом документе
необходима провести
проверку правописания.
С другой стороны, ошибка
в слове - не конец света.
Остальная часть абзаца
не требует изменений.
Новый текст можно
добавлять в конец документа.
\end{verbatim}

\textbf{f2.txt}
\begin{verbatim}
Это важное замечание!
Поэтому оно должно
быть расположено
в начале этого
документа!

Эта часть документа
оставалась неизменной
от версии к версии. Если
в ней нет изменений, она
не должна отображаться.
Иначе это не способствует
выводу оптимального
объёма информации.

В этом документе
необходимо провести
проверку правописания.
С другой стороны, ошибка
в слове - не конец света.
Остальная часть абзаца
не требует изменений.
Новый текст можно
добавлять в конец документа.

Этот абзац содержит
важные дополнения
для данного документа.
\end{verbatim}

Запустим для них утилиту:

\begin{verbatim}
fallfire13@DESKTOP-M7F3IHA:~/DA_kp$ time ./a.out f1.txt f2.txt
+ Это важное замечание!
+ Поэтому оно должно
+ быть расположено
+ в начале этого
+ документа!
+
  Эта часть документа
  оставалась неизменной
  от версии к версии. Если
  в ней нет изменений, она
  не должна отображаться.
  Иначе это не способствует
  выводу оптимального
- объёма произведённых
- изменений.
+ объёма информации.

- Этот абзац содержит
- устаревший текст.
- Он будет удалён
- в ближайшем будущем.
-
  В этом документе
- необходима провести
+ необходимо провести
  проверку правописания.
  С другой стороны, ошибка
  в слове - не конец света.
  Остальная часть абзаца
  не требует изменений.
  Новый текст можно
  добавлять в конец документа.
+
+ Этот абзац содержит
+ важные дополнения
+ для данного документа.

real    0m0.027s
user    0m0.000s
sys     0m0.000s
\end{verbatim}

Проверим работу, запустив стандартную утилиту diff:

\begin{verbatim}
fallfire13@DESKTOP-M7F3IHA:~/DA_kp$ time diff f1.txt f2.txt
0a1,6
> Это важное замечание!
> Поэтому оно должно
> быть расположено
> в начале этого
> документа!
>
8,14c14
< объёма произведённых
< изменений.
<
< Этот абзац содержит
< устаревший текст.
< Он будет удалён
< в ближайшем будущем.
---
> объёма информации.
17c17
< необходима провести
---
> необходимо провести
24a25,28
>
> Этот абзац содержит
> важные дополнения
> для данного документа.

real    0m0.019s
user    0m0.000s
sys     0m0.000s
\end{verbatim}

Как видим, результаты работ совпадают.

\textbf{Исследование времени выполнения}

\begin{verbatim}
    fallfire13@DESKTOP-M7F3IHA:~/DA_kp$ time ./a.out f1.txt f1.txt
  Эта часть документа
  оставалась неизменной
  от версии к версии. Если
  в ней нет изменений, она
  не должна отображаться.
  Иначе это не способствует
  выводу оптимального
  объёма произведённых
  изменений.

  Этот абзац содержит
  устаревший текст.
  Он будет удалён
  в ближайшем будущем.

  В этом документе
  необходима провести
  проверку правописания.
  С другой стороны, ошибка
  в слове - не конец света.
  Остальная часть абзаца
  не требует изменений.
  Новый текст можно
  добавлять в конец документа.

real    0m0.017s
user    0m0.000s
sys     0m0.000s
\end{verbatim}
Построение diff на одинаковых файлах быстрое, т.к. алгоритм останавливается за один шаг.

\begin{verbatim}
fallfire13@DESKTOP-M7F3IHA:~/DA_kp$ tac f1.txt > f3.txt
fallfire13@DESKTOP-M7F3IHA:~/DA_kp$ time ./a.out f1.txt f3.txt
- Эта часть документа
- оставалась неизменной
- от версии к версии. Если
- в ней нет изменений, она
- не должна отображаться.
- Иначе это не способствует
- выводу оптимального
- объёма произведённых
- изменений.
+ добавлять в конец документа.
+ Новый текст можно
+ не требует изменений.
+ Остальная часть абзаца
+ в слове - не конец света.
+ С другой стороны, ошибка
+ проверку правописания.
+ необходима провести
+ В этом документе

- Этот абзац содержит
- устаревший текст.
- Он будет удалён
  в ближайшем будущем.
+ Он будет удалён
+ устаревший текст.
+ Этот абзац содержит

- В этом документе
- необходима провести
- проверку правописания.
- С другой стороны, ошибка
- в слове - не конец света.
- Остальная часть абзаца
- не требует изменений.
- Новый текст можно
- добавлять в конец документа.
+ изменений.
+ объёма произведённых
+ выводу оптимального
+ Иначе это не способствует
+ не должна отображаться.
+ в ней нет изменений, она
+ от версии к версии. Если
+ оставалась неизменной
+ Эта часть документа

real    0m0.076s
user    0m0.000s
sys     0m0.000s
\end{verbatim}

Сложность алгоритма очень схожа с $O$(($N$+$M$)$D$). Особенно это видно на примере с $tac$, где из-за того, что строки файла в обратном порядке, требуется заменить все строки, и глубина $D$=$N$+$M$.

\pagebreak