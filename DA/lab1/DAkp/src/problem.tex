\section{Постановка задачи}

\textbf{Задание}

Разработать на языке C++ программу, способную выводить наименьшее количество действий, позволяющих преобразовать один файл в другой (утилита diff).

\textbf{Метод решения}

Самым простом алгоритмом, выводящим доступный для человска формат, является
алгоритм \textbf{Юджина Майерса}, поверхностно описать который можно так:

Проблему нахождения наименьшего изменения между двумя файлами $A$ и $B$ раз-
меров $N$ и $M$ одновременно можно рассматривать как проблему нахождения пути на графе размером $N$ x $M$ между вершинами (0,0) и ($N$,$M$), причем на этом графе:

\begin{itemize}
    \item Горизонтальная грань ($x$,$y$) $\rightarrow$ ($x$+1,$y$) означает удаление в файле $A$ строки $x$ (в файлах индексация строк с счипается единицы);
    \item Вертикальная грань ($x$,$y$) $\rightarrow$ ($x$,$y$+1) означает добавление из файла $B$ строки $y$;
    \item Диагональная грань ($x$,$y$) $\rightarrow$ ($x$+1,$y$+1) означает, что строки $x$ и $y$ в файлах $A$ и $B$ соответственно равны.
\end{itemize}

Если считать, что у горизонтальных и вертикальных граней вес равен единице, а у
диагональных - нулю, то задача решаема алгоритмом Дейкстры, но он не всегда будет
выводить читабельный diff-вывод.

Алгоритм Майерса пользуется тем, что путь в графе всегда идет из верхнего левого
угла в нижний правый угол.

Введем понятие уровня и D-пути:
\begin{itemize}
    \item Уровень $k$ на графе - это номер диагонали по сравнению с диагональю, на которой
лежит точка (0,0). Соответственно, диагональ с точкой (0,0) имеет уровень $k$ = 0, диагональ над ней - $k$ = 1, диагональ под ней - $k$ = -1 и т.д. Уровень можно высчитать по формуле $k$ = $x$ - $y$.
    \item D-путь -  (D - 1)-путь, после которого идет или горизонтальная грань, или вертикальная грань, причем 0-путь - это путь, состоящий только из этой грани. После этой грани существует возможно пустая последовательность диагоналей.
\end{itemize}

Алгоритм имеет максимум $N$+$M$ повторений, причем на $d$-ом повторении идет удлинение всех $d$-путей, лежащих на уровнях -$d$, -$d$+2,..., $d$-2, $d$. Удлинение на уровне $k$ идет за удлинения пути либо на $k$-1 уровне или на $k$+1, в зависимости от того, какой длиннее. Приоритет бкдет отдан $k$-1 уровню когда это возможно, т.к. он удлиняется за счёт горизонтальной грани (т.е. удаления из $A$). Первый $d$-путь, достигший точки ($N$,$M$), считается оптимальным.

\pagebreak
