\section{Выводы}

Кроме своего стандартного использования (сравнения файлов), можно diff ещё можно использовать для:
\begin{itemize}
    \item Построение patch-файлов, Они могут понадобиться для изменения конкретных частей файла без изменения его всего (в отличии от передачи всего файла).
    \item Минимальная передача данных при синхронизации бинарных файлов за счет проверки их на одинаковость и отправки только отличий (\textbf{rsync}).
    \item Противу интуитивным понятиям, diff можно обобщить до сравнения всего, что можно редактировать, в том числе двух бинарных файлов.
    \item Более специфично: обнаружение и сравнение мутации ДНК.
\end{itemize}

Стоит заметить, для большинства вводов существует более чем одна минимальная разность файлов, но одна разность может иметь раскинутые действия по всему файлу, а другая будет иметь сгруппированные в одном месте действия. Далее рассматриваются четыре алгоритма, встроенные в \textbf{git diff}:

\begin{enumerate}
    \item Алгоритм \textbf{Юджина Майерса} в линейном пространстве - стандартный применяемый алгоритм, используемый при вызове утилиты. Чаще всего выводит хорошие результаты за быстрое время и малую память, но на пекоторых вводах запинается и выводит сильно смешанный вывод. Другой флаг $minimal$ работает на том же алгоритме, но рассматривает больше промежуточных вариантов для вывода чего-то более читабельного. Можно применять, когда нет причин использовать другие алгоритмы.
    \item Алгоритм \textbf{patience} - алторитм делит файл на секции, используя общие строки, которые не повторяются нигде в самих документах. Деление файлов на общие секции и поиск изменений в самих секциях выводит сгруппированную разницу чаще, чем у Майерса. Пространственная сложность линенная.
    \item Алгоритм \textbf{histogram} - является модификацией patience, превосходящий по скорости и результатам и Майерса и patience. В отличие от patience, ищет не уникальные элементы, а наименее повторяющиеся. Следует применять при сравнении исходного кода.
\end{enumerate}

Один из неупомянутых алгоритмов (который был введён столько раз,что уже неизвестно, кто первым его придумал): алгоритм Вагнера-Фишера, который считает рсстояния между каждым префиксом обоих массивов, что имеет предположительно пространственную и алгоритмическую сложность $O$($NM$).