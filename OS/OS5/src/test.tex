\section{Пример работы}

Продемонстрирую процесс сборки программ:

\begin{verbatim}

fallfire13@DESKTOP-M7F3IHA:~/OS_labs/OS5$ make
gcc -Wall -Wextra -c -I./include src/main_runt.c -o bin/main_runt.o
gcc -Wall -Wextra -c -fpic -I./include src/pi_1.c -o bin/pi_1.o
gcc -shared bin/pi_1.o -o bin/libpi_1.so
gcc -Wall -Wextra -c -fpic -I./include src/square_1.c -o bin/square_1.o
gcc -shared bin/square_1.o -o bin/libsquare_1.so
gcc -Wall -Wextra -c -fpic -I./include src/pi_2.c -o bin/pi_2.o
gcc -shared bin/pi_2.o -o bin/libpi_2.so
gcc -Wall -Wextra -c -fpic -I./include src/square_2.c -o bin/square_2.o
gcc -shared bin/square_2.o -o bin/libsquare_2.so
gcc bin/main_runt.o -o main_runt -ldl
gcc -Wall -Wextra -c -I./include src/main_link.c -o bin/main_link.o
gcc -L./bin -Wl,-rpath=./bin bin/main_link.o bin/libpi_1.so bin/libsquare_1.so -o main_link -lpi_1 -lsquare_1

\end{verbatim}

Работа первой программы:

\begin{verbatim}

fallfire13@DESKTOP-M7F3IHA:~/OS_labs/OS5$ ./main_link
0
no such command
1 1
pi = 4.000000
1 10
pi = 3.041840
1 30
pi = 3.108268
1 100
pi = 3.131593
2 2 3
square = 6.000000
2 1.5 3
square = 4.500000

\end{verbatim}

\pagebreak

Работа второй программы:

\begin{verbatim}

fallfire13@DESKTOP-M7F3IHA:~/OS_labs/OS5$ ./main_runt
1 1
pi = 4.000000
1 10
pi = 3.041840
1 30
pi = 3.108268
1 100
pi = 3.131593
2 2 3
square = 6.000000
2 1.5 3
square = 4.500000
0
implementation changed to 2
1 1
pi = 4.000000
1 10
pi = 3.002177
1 30
pi = 3.091339
1 100
pi = 3.126081
2 2 3
square = 3.000000
2 1.5 3
square = 2.250000

\end{verbatim}

\pagebreak


