\section{Общие сведения о программе}

Лабораторная состоит из двух программ. В первом случае мы подключаем библиотеку на этапе линковки, как мы обычно подключаем math библиотеку в Си. Во втором случае нам необходимо подключить shared библиотеку в рантайме при помощи системных вызовов dlopen, dlsym, dlclose. Для того, чтобы скомпилировать shared библиотеки, нам необходимо использовать особые ключи компилятора.

\pagebreak

\section{Общий метод и алгоритм решения}

Для реализации поставленной задачи необходимо:

\begin{enumerate}
    \item Изучить принципы создания и использования динамических библиотек
    \item Написать функции библиотеки
    \item Написать первую программу
    \item Написать вторую программу, используя системные вызовы
    \item Собрать проект
    \item Написать тесты
\end{enumerate}

\pagebreak

\section{Исходный код}

{\large Функции библиотек}

\textbf{pi\_1.c}

\begin{lstlisting}[language=C]

#include "pi.h"

float pi(int n) {
    if (n < 1) {
        return 0;
    }
    float res = 1;
    int sign = -1;
    for (int i = 1; i < n; ++i) {
        res += sign * ((float) 1 / (1 + i*2));
        sign *= -1;
    }
    return res * 4;
}

\end{lstlisting}

\textbf{pi\_2.c}

\begin{lstlisting}[language=C]

#include "pi.h"

float pi(int n) {
    if (n < 1) {
        return 0;
    }
    float res = 1;
    for (int i = 1; i <= n; ++i) {
        float a = 2 * ((i + 1)/2);
        float b = 1 + 2 * (i/2);
        res *= a / b;
    }
    return res * 2;
}

\end{lstlisting}

\textbf{square\_1.c}

\begin{lstlisting}[language=C]

#include "square.h"

float square(float a, float b) {
    return a * b;
}

\end{lstlisting}

\pagebreak

\textbf{square\_2.c}

\begin{lstlisting}[language=C]

#include "square.h"

float square(float a, float b) {
    return a * b / 2;
}

\end{lstlisting}

{\large Первая программа}

\textbf{main\_link.c}

\begin{lstlisting}[language=C]

#include <stdio.h>
#include <string.h>
#include <ctype.h>
#include <stdlib.h>

#include "pi.h"
#include "square.h"

#define MAX_INPUT_LEN 256

int main() {
    char input[MAX_INPUT_LEN + 1];
    while (fgets(input, MAX_INPUT_LEN + 1, stdin) != NULL) {
        char dup[MAX_INPUT_LEN + 2];
        strcpy(dup + 1, input);
        dup[0] = '\0';
        size_t len = strlen(input);
        int argc = 0;
        for (size_t i = 1; i < len; ++i) {
            if (!isspace(dup[i]) && isspace(dup[i+1])) {
                ++argc;
            }
        }
        char *argv[argc];
        int cur = 0;
        for (size_t i = 1; i < len + 1; ++i) {
            if (isspace(dup[i])) {
                dup[i] = '\0';
            }
            else if (dup[i-1] == '\0') {
                argv[cur] = &dup[i];
                ++cur;
            }
        }
        if (argc == 0) {
            continue;
        }
        if (strcmp(argv[0], "1") == 0) {
            if (argc != 2) {
                fprintf(stderr, "bad arguments\n");
                continue;
            }
            int n = atoi(argv[1]);
            printf("pi = %f\n", pi(n));
        }
        else if (strcmp(argv[0], "2") == 0) {
            if (argc != 3) {
                fprintf(stderr, "bad arguments\n");
                continue;
            }
            float a = atof(argv[1]);
            float b = atof(argv[2]);
            printf("square = %f\n", square(a, b));
        }
        else {
            fprintf(stderr, "no such command\n");
        }
    }

    return 0;
}

\end{lstlisting}

\pagebreak

{\large Вторая программа}

\textbf{main\_runt.c}

\begin{lstlisting}[language=C]

#include <stdio.h>
#include <string.h>
#include <ctype.h>
#include <stdlib.h>
#include <dlfcn.h>

#define MAX_INPUT_LEN 256

int main() {
    int cur_impl = 1;
    void *pi_1_h = dlopen("./bin/libpi_1.so", RTLD_NOW);
    void *pi_2_h = dlopen("./bin/libpi_2.so", RTLD_NOW);
    void *square_1_h = dlopen("./bin/libsquare_1.so", RTLD_NOW);
    void *square_2_h = dlopen("./bin/libsquare_2.so", RTLD_NOW);
    if (pi_1_h == NULL || pi_2_h == NULL ||
        square_1_h == NULL || square_2_h == NULL) 
    {
        perror("error");
        exit(1);
    }
    float (*pi)(int) = dlsym(pi_1_h, "pi");
    float (*square)(float, float) = dlsym(square_1_h, "square");
    char input[MAX_INPUT_LEN + 1];
    while (fgets(input, MAX_INPUT_LEN + 1, stdin) != NULL) {
        char dup[MAX_INPUT_LEN + 2];
        strcpy(dup + 1, input);
        dup[0] = '\0';
        size_t len = strlen(input);
        int argc = 0;
        for (size_t i = 1; i < len; ++i) {
            if (!isspace(dup[i]) && isspace(dup[i+1])) {
                ++argc;
            }
        }
        char *argv[argc];
        int cur = 0;
        for (size_t i = 1; i < len + 1; ++i) {
            if (isspace(dup[i])) {
                dup[i] = '\0';
            }
            else if (dup[i-1] == '\0') {
                argv[cur] = &dup[i];
                ++cur;
            }
        }
        if (argc == 0) {
            continue;
        }
        if (strcmp(argv[0], "0") == 0) {
            if (argc != 1) {
                fprintf(stderr, "bad arguments\n");
                continue;
            }
            if (cur_impl == 1) {
                pi = dlsym(pi_2_h, "pi");
                square = dlsym(square_2_h, "square");
                cur_impl = 2;
            }
            else {
                pi = dlsym(pi_1_h, "pi");
                square = dlsym(square_1_h, "square");
                cur_impl = 1;
            }
            printf("implementation changed to %d\n", cur_impl);
        }
        else if (strcmp(argv[0], "1") == 0) {
            if (argc != 2) {
                fprintf(stderr, "bad arguments\n");
                continue;
            }
            int n = atoi(argv[1]);
            printf("pi = %f\n", pi(n));
        }
        else if (strcmp(argv[0], "2") == 0) {
            if (argc != 3) {
                fprintf(stderr, "bad arguments\n");
                continue;
            }
            float a = atof(argv[1]);
            float b = atof(argv[2]);
            printf("square = %f\n", square(a, b));
        }
        else {
            fprintf(stderr, "no such command\n");
        }
    }
    dlclose(pi_1_h);
    dlclose(pi_2_h);
    dlclose(square_1_h);
    dlclose(square_2_h);

    return 0;
}

\end{lstlisting}

\pagebreak