\section{Постановка задачи}

{\bfseries Цель работы:} 

Приобретение практических навыков в:

\begin{itemize}
    \item Создание динамических библиотек
    \item Создание программ, которые используют функции динамических библиотек
\end{itemize}

{\bfseries Задание (вариант 28):} 

Требуется создать динамические библиотеки, которые реализуют определенный функционал. Далее использовать данные библиотеки 2-мя способами:

\begin{enumerate}
    \item Во время компиляции (на этапе «линковки»/linking)
    \item Во время исполнения программы, загрузив библиотеки в память с помощью системных вызовов
\end{enumerate}

В конечном итоге, в лабораторной работе необходимо получить следующие части:

\begin{itemize}
    \item Динамические библиотеки, реализующие контракты, которые заданы вариантом;
    \item Тестовая программа (программа No1), которая используют одну из библиотек, используя знания полученные на этапе компиляции;
    \item Тестовая программа (программа No2), которая загружает библиотеки, используя только их местоположение и контракты.
\end{itemize}

Провести анализ двух типов использования библиотек. Пользовательский ввод для обоих программ должен быть организован следующим образом:

\begin{enumerate}
    \item Если пользователь вводит команду «0», то программа переключает одну реализацию контрактов на другую (необходимо только для программы No2). Можно реализовать лабораторную работу без данной функции, но максимальная оценка в этом случае будет «хорошо»;
    \item «1 arg1 arg2 ... argN», где после «1» идут аргументы для первой функции, предусмотренной контрактами. После ввода команды происходит вызов первой функции, и на экране появляется результат её выполнения;
    \item «2 arg1 arg2 ... argM», где после «2» идут аргументы для второй функции, предусмотренной контрактами. После ввода команды происходит вызов второй функции, и на экране появляется результат её выполнения.
\end{enumerate}

\textbf{Вариант 28} 
\begin{enumerate}
    \item Рассчет значения числа Пи при заданной длине ряда\\
\textbf{Реализация 1} Ряд Лейбница\\
\textbf{Реализация 2} Формула Валлиса
\item Подсчет площади плоской геометрической фигуры по двум сторонам\\
\textbf{Реализация 1} Фигура прямоугольник\\
\textbf{Реализация 2} Фигура прямоугольный треугольник
\end{enumerate}

\pagebreak
