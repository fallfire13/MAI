\section{Общие сведения о программе}

Программа компилируется из файла main.c. В программе используются следующие системные вызовы:

\begin{enumerate}
    \item \textbf{pipe} –– принимает массив из двух целых чисел, в случае успеха массив будет содержать два файловых дескриптора, которые будут использоваться для конвейера, первое число в массиве предназначено для чтения, второе для записи, а так же вернется 0. В случае неуспеха вернется -1.
    \item \textbf{fork} –– создает новый процесс, который является копией родительского процесса, за исключением разных pid. В случае успеха fork() возвращает 0 для ребенка, число больше 0 для родителя – pid ребенка, в случае ошибки возвращает -1.
    \item \textbf{open} –– создает или открывает файл, если он был создан. В качестве аргументов принимает путь до файла, режим доступа (запись, чтение и т.п.),  модификатор доступа ( при создании можно указать права для файла ). Возвращает в случае успеха файловый дескриптор – положительное число, иначе возвращает -1.
    \item \textbf{close} –– принимает файловый дескриптор в качестве аргумента, удаляет файловый дескриптор из таблицы дескрипторов, в случае успеха вернет 0, в случае неуспеха вернет -1.
    \item \textbf{read} –– предназначена для чтения какого-то числа байт из файла, принимает в качестве аргументов файловый дескриптор, буфер, в который будут записаны данные и число байт. В случае успеха вернет число прочитанных байт, иначе -1.
    \item \textbf{write} –– предназначена для записи какого-то числа байт в файл, принимает в качестве аргументов файловый дескриптор, буфер, из которого будут считаны данные для записи и число байт. В случае успеха вернет число записанных байт, иначе -1.
    \item \textbf{prctl} -- манипулирует аспектами поведения родительского процесса или потока. Принимает аспект поведения в виде числа, далее идут varargs, количество и значение которых зависит от первого аргумента.
\end{enumerate}

\pagebreak

\section{Общий метод и алгоритм решения}

Для реализации поставленной задачи необходимо:

\begin{enumerate}
    \item Изучить принципы работы pipe и fork.
    \item Создать каналы связи для каждого из дочерних процессов
    \item Написать код для создания дочерних процессов
    \item Написать функцию работы родительского процесса
    \item Написать функцию работы дочернего процесса
    \item Написать обработку ошибок
    \item Написать тесты
\end{enumerate}

\pagebreak

\section{Исходный код}

\textbf{main.c}

\begin{lstlisting}[language=C]

#include <stdio.h>
#include <unistd.h>
#include <string.h>
#include <stdlib.h>
#include <signal.h>
#include <sys/prctl.h>

#include <fcntl.h>
#include <sys/stat.h>

#define PATH_MAX 4096
#define INPUT_BUFFER 4096
#define MAX_PROCESSES 2
#define READ 0
#define WRITE 1
#define BOUNDARY 10

#define STDIN 0

size_t my_read(char *buff, size_t max_bytes, int fd) {
    char temp;
    size_t i;
    for (i = 0; i < max_bytes - 1; ++i) {
        if (read(fd, &temp, sizeof(char)) == 0 || temp == '\n') {
            break;
        }
        buff[i] = temp;
    }
    buff[i] = '\0';
    return i;
}

void parent_death(int sig) {
    exit(0);
}

void parentjob(int fd[MAX_PROCESSES][2]) {
    for (int i = 0; i < MAX_PROCESSES; i++) {
        close(fd[i][READ]);
    }
    char filename[PATH_MAX + 1];
    for (int i = 0; i < MAX_PROCESSES; i++) {
        //fgets(filename, PATH_MAX + 1, stdin);
        my_read(filename, PATH_MAX + 1, STDIN);
        write(fd[i][WRITE], filename, PATH_MAX + 1);
    }
    char input[INPUT_BUFFER + 1];
    //while (fgets(input, INPUT_BUFFER + 1, stdin) != NULL) {
    while (my_read(input, INPUT_BUFFER + 1, STDIN) != 0) {
        if (strlen(input) - 1 <= BOUNDARY) {
            write(fd[0][WRITE], input, INPUT_BUFFER + 1);
        }
        else {
            write(fd[1][WRITE], input, INPUT_BUFFER + 1);
        }
    }
}

void childjob(int fd[2], pid_t parent_pid) {
    prctl(PR_SET_PDEATHSIG, SIGTERM);
    signal(SIGTERM, parent_death);
    if (getppid() != parent_pid) {
        parent_death(SIGTERM);
    }
    close(fd[WRITE]);
    char filename[PATH_MAX + 1];
    read(fd[READ], filename, PATH_MAX + 1);
    //FILE *fp = fopen(filename, "w");
    int file_des = open(filename, O_WRONLY | O_CREAT, S_IWUSR | S_IRUSR);
    char input[INPUT_BUFFER + 1];
    while (1) {
        read(fd[READ], input, INPUT_BUFFER + 1);
        size_t input_len = strlen(input);
        for (int i = 0; i < input_len / 2; i++) {
            char temp = input[i];
            input[i] = input[input_len - 1 - i];
            input[input_len - 1 - i] = temp;
        }
        //fputs(input, fp);
        write(file_des, input, strlen(input));
        write(file_des, "\n", 1);
    }
}

int main() {
    int fd[MAX_PROCESSES][2];
    for (int i = 0; i < MAX_PROCESSES; i++) {
        pipe(fd[i]);
    }
    pid_t parent_pid = getpid();
    pid_t temp_pid = 1;
    unsigned int id = 0;
    for (int i = 0; i < MAX_PROCESSES; i++) {
        if (temp_pid != 0) {
            id = i;
            temp_pid = fork();
            if (temp_pid == -1) {
                perror("fork error");
                exit(1);
            }
        }
        else {
            break;
        }
    }
    if (temp_pid == 0) {
        childjob(fd[id], parent_pid);
    }
    else {
        parentjob(fd);
    }
    
    return 0;
}

\end{lstlisting}

\pagebreak