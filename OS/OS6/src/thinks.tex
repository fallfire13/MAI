\section{Вывод}

При написании 6-8 лабораторной работы я ознакомился с принципом действия ZeroMQ и очередей сообщений, хоть и не познал даннную технологию в полной мере. Использовать ее оказалось намного удобнее, чем стандартные сокеты, на которых она построена. Однако же сама оригинальная документация библиотеки оказалась не слишком удобна, как мне показалось, она рассчитана на человека, который уже имел дело с данной сферой IT. Лабораторную пришлось писать на C++ в главную очередь из-за наличия STL и наличия умных указателей и деструкторов в частности. Это сильно облегчило работу с памятью. В ходе написания лабораторной работы я улучшил свой навык разделения задачи на подзадачи, а также навык разбиения на простые для понимания абстракции.

