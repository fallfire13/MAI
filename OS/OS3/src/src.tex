\section{Общие сведения о программе}

Матрицы хранятся в структуре Matrice. На вход программе подается размер первой матрицы, первая матрица, затем размер второй матрицы, вторая матрица. Умножение происходит по стандартной формуле умножения матриц. Ячейки новой матрицы для вычиления распределяются по потокам равномерно. После ввода двух матриц программа производит умножение и выводит результирующую матрицу.

\pagebreak

\section{Общий метод и алгоритм решения}

Для реализации поставленной задачи необходимо:

\begin{enumerate}
    \item Изучить принципы работы pthread.
    \item Написать структуру для хранения матрицы
    \item Написать функцию ввода матрицы
    \item Написать функцию вывода матрицы
    \item Написать функцию для умножения матриц в одном потоке (для бенчмарка)
    \item Написать распределение ячеек новой матрицы по потокам
    \item Написать функцию вычисления ячеек новой матрицы в отдельном потоке
    \item Написать бенчмарк
    \item Написать обработку ошибок
    \item Написать тесты
\end{enumerate}

\pagebreak

\section{Исходный код}

\textbf{main.c}

\begin{lstlisting}[language=C]

#include <complex.h>
#include <stdio.h>
#include <stdlib.h>
#include <pthread.h>
#include <time.h>

#define SECOND2NANO 1000000000

typedef struct {
    size_t width;
    size_t height;
    double complex **buff;
} Matrice;

typedef struct {
    Matrice *lhs;
    Matrice *rhs;
    Matrice *result;
    size_t linear_start;
    size_t linear_end;
} _Matrice_params;

void matrice_fill(Matrice *subj) {
    scanf("%zu %zu", &subj->height, &subj->width);
    subj->buff = malloc(subj->height * sizeof(double complex *));
    for (size_t i = 0; i < subj->height; ++i) {
        subj->buff[i] = malloc(subj->width * sizeof(double complex));
        for (size_t j = 0; j < subj->width; ++j) {
            double temp_real;
            double temp_imag;
            scanf("%lf;%lf", &temp_real, &temp_imag);
            subj->buff[i][j] = CMPLX(temp_real, temp_imag);
        }
    }
}

void matrice_print(const Matrice subj) {
    for (size_t i = 0; i < subj.height; ++i) {
        for (size_t j = 0; j < subj.width; ++j) {
            printf("%.2f;%.2f\t", creal(subj.buff[i][j]), cimag(subj.buff[i][j]));
        }
        printf("\n");
    }
}

void matrice_free(Matrice subj) {
    for (size_t i = 0; i < subj.height; ++i) {
        free(subj.buff[i]);
    }
    free(subj.buff);
}

size_t _get_2d_x(const Matrice *matrice, size_t linear_coord) {
    return linear_coord % matrice->width;
}

size_t _get_2d_y(const Matrice *matrice, size_t linear_coord) {
    return linear_coord / matrice->width;
}

void *_matrice_indiv_thread(void *params_void) {
    _Matrice_params *params = (_Matrice_params *) params_void;
    for (size_t i = params->linear_start; i < params->linear_end; ++i) {
        size_t x = _get_2d_x(params->result, i);
        size_t y = _get_2d_y(params->result, i);
        params->result->buff[y][x] = CMPLX(0, 0);
        for (size_t j = 0; j < params->lhs->width; ++j) {
            params->result->buff[y][x] += params->lhs->buff[y][j] * params->rhs->buff[j][x];
        }      
    }
    return NULL;
}

Matrice matrice_mult_threads(Matrice lhs, Matrice rhs, unsigned int threads_limit) {
    if (lhs.width != rhs.height) {
        printf("inappropriate matrices' sizes\n");
        exit(1);
    }

    Matrice result;
    result.height = lhs.height;
    result.width = rhs.width;
    result.buff = malloc(result.height * sizeof(double complex *));
    for (size_t i = 0; i < result.height; ++i) {
        result.buff[i] = malloc(result.width * sizeof(double complex));
    }

    size_t linear_size = result.height * result.width;
    if (threads_limit > linear_size) {
        threads_limit = linear_size;
    }
    size_t quotient = linear_size / threads_limit;
    size_t remainder = linear_size % threads_limit;
    size_t linear_iter = 0;
    pthread_t threads[threads_limit];
    _Matrice_params params[threads_limit];

    for (unsigned int i = 0; i < threads_limit; ++i) {
        params[i].lhs = &lhs;
        params[i].rhs = &rhs;
        params[i].result = &result;
        params[i].linear_start = linear_iter;
        linear_iter += quotient;
        if (remainder > 0) {
            ++linear_iter;
            --remainder;
        }
        params[i].linear_end = linear_iter;
        if (pthread_create(&threads[i], NULL, _matrice_indiv_thread,  &params[i]) != 0) {
            printf("error with thread creating occured\n");
            exit(EXIT_FAILURE);
        }
    }
    for (unsigned int i = 0; i < threads_limit; ++i) {
        pthread_join(threads[i], NULL);
    }
    return result;
}

Matrice matrice_mult_casual(Matrice lhs, Matrice rhs) {
    Matrice result;
    result.height = lhs.height;
    result.width = rhs.width;
    result.buff = malloc(result.height * sizeof(double complex *));
    for (size_t i = 0; i < result.height; ++i) {
        result.buff[i] = malloc(result.width * sizeof(double complex));
    }

    for (size_t y = 0; y < result.height; ++y) {
        for (size_t x = 0; x < result.width; ++x) {
            result.buff[y][x] = 0;
            for (size_t i = 0; i < lhs.width; ++i) {
                result.buff[y][x] += lhs.buff[y][i] * rhs.buff[i][x];
            }
        }
    }

    return result;
}

int main(int argc, char **argv) {
    if (argc != 2 || atoi(argv[1]) == 0) {
        printf("bad arguments\n");
        exit(EXIT_FAILURE);
    }
    unsigned int threads_limit = atoi(argv[1]);
    Matrice lhs;
    matrice_fill(&lhs);
    Matrice rhs;
    matrice_fill(&rhs);
    Matrice result;

    struct timespec casual_start, casual_end;
    timespec_get(&casual_start, TIME_UTC);
    result = matrice_mult_casual(lhs, rhs);
    timespec_get(&casual_end, TIME_UTC);
    matrice_free(result);
    struct timespec thread_start, thread_end;
    timespec_get(&thread_start, TIME_UTC);
    result = matrice_mult_threads(lhs, rhs, threads_limit);
    timespec_get(&thread_end, TIME_UTC);

    fprintf(stderr, "Casual multiplying: %lf\n", ((casual_end.tv_sec * SECOND2NANO + casual_end.tv_nsec) - 
                                                 (casual_start.tv_sec * SECOND2NANO + casual_start.tv_nsec)) / (double) SECOND2NANO);
    fprintf(stderr, "Threading multiplying: %lf\n", ((thread_end.tv_sec * SECOND2NANO + thread_end.tv_nsec) - 
                                                     (thread_start.tv_sec * SECOND2NANO + thread_start.tv_nsec)) / (double) SECOND2NANO);
    matrice_print(result);

    matrice_free(lhs);
    matrice_free(rhs);
    matrice_free(result);

    return 0;
}

\end{lstlisting}

\pagebreak