\section{Постановка задачи}

\textbf{Цель курсового проекта}

\begin{enumerate}
    \item Приобретение практических навыков в использовании знаний, полученных в течении курса
    \item Проведение исследования в выбранной предметной области
\end{enumerate}

\textbf{Задание}

Необходимо спроектировать и реализовать программный прототип в соответствии с выбранным
вариантом. Произвести анализ и сделать вывод на основании данных, полученных при работе
программного прототипа.

\textbf{Вариант на \enquote{удовлетворительно}.}

Необходимо написать три программы. Далее будем обозначать эти программы A, B, C.
Программа A принимает из стандартного потока ввода строки, а далее их отправляет программе
С. Отправка строк должна производится построчно. Программа C печатает в стандартый вывод,
полученную строку от программы A. После получения программа C отправляет программе А
сообщение о том, что строка получена. До тех пор пока программа А не примет «сообщение о
получение строки» от программы С, она не может отправялять следующую строку программе С.
Программа B пишет в стандартный вывод количество отправленных символов программой А и
количество принятых символов программой С. Данную информацию программа B получает от
программ A и C соответственно.
Способ организация межпроцессного взаимодействия выбирает студент.

\pagebreak
