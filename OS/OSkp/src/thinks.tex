\section{Вывод}

В процессе работы над данным курсовым проектом я освежил основы межпроцессного взаимодействия с использованием технологии pipe и сигналов. Pipe является несложной технологией для межпроцессного взаимодействия и предлагает удобный интерфейс.

Также я выяснил более полно, как работает $execl()$ - системный вызов замены образа процесса на другую программу - и таблицы файловых дескрипторов у процессов. Использование $execl()$ упрощает написание программы, однако выяснил я это намного позже написания второй лабораторной работы, в которой обе программы были написаны в одном файле. Разделение кода на подобные логические единицы упрощает разработку и понимание кода. 

