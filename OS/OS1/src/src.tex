\section{Общий метод и алгоритм решения}

Для диагностики работы программы будем использовать утилиту strace на операционной системе Ubuntu.
Чтобы правильно запустить утилиту, нужно:

\begin{itemize}
    \item Указать исполняемый файл
    \item Указать ключ -f, чтобы отслеживать дочерние процессы
    \item Указать ключ -o, чтобы вывод strace шел в файл
\end{itemize}

Вывод утилиты состоит из названия системных вызовов, переданных им аргументов и возвращаемых ими значений.

В данном случае используются следующие основные утилиты:

\begin{itemize}
    \item \textbf{execve} -- выполняет программу, заданную параметром filename.
    \item \textbf{mmap, munmap} - отражает файлы или устройства в памяти или снимает их отражение.
    \item \textbf{openat} -- открытие  файла, если открытие происходит успешно, присваивается файловый дескриптор.
    \item \textbf{close} -- закрытие файла, если закрытие происходит успешно, возвращается 0.
    \item \textbf{read} -- чтение из файла, возвращает количество считанных байт.
    \item \textbf{write} -- запись в файл, возвращает количество записанных байт.
\end{itemize}

\pagebreak
